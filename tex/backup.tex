\begin{frame}{Experimental Questions}
\begin{itemize}
\item how do genetic regulatory networks evolved with direct plasticity differ structurally from control networks? \cite{Reisinger2007AcquiringRepresentations}
\item impact of other modes of direct plasticity on evolvability (rule noise, fixed states, intermediate state perturbation)?
\item impact of indirect plasticity on evolvability?   
\item combined impact of direct and indirect plasticity on evolvability?   
\end{itemize}
\end{frame}


\begin{frame}{Generating and Reading an Evolvability Signature}
  \input{figs/generating_reading_evolvability_signature.tex}
\end{frame}

\begin{frame}{Evolvability Visualization $W=0$} 
\begin{figure}
    \centering
    \includegraphics[width=0.8\textwidth]{img/ev_w0}
 	\captionsetup{singlelinecheck=off,justification=raggedright}
  	\caption{Evolvability visualization of champions evolved with only a primary condition/objective pair.}
    \label{fig:ev_w0}
\end{figure}
\end{frame}

\begin{frame}{Evolvability Visualization $W=0.2$}
\begin{figure}
    \centering
    \includegraphics[width=0.8\textwidth]{img/ev_w0_2}
 	\captionsetup{singlelinecheck=off,justification=raggedright}
  	\caption{Evolvability visualization of champions evolved with primary and secondary condition/objective pairs.}
    \label{fig:es_w0_2}
\end{figure}
\end{frame}


\begin{frame}{Environmental Influence on the Phenotype}
\begin{itemize}
	\item in biology, genotype not sole determinant of phenotype
    \item $P = G + E$
    \item plasticity: phenotypic response to the environment
    \item direct plasticity versus indirect plasticity
\end{itemize}
\end{frame}

\begin{frame}{Direct Plasticity: Biological Intuition}
  \input{figs/elephant_developmental_perturbation.tex}
\end{frame}

\begin{frame}{Indirect Plasticity: Biological Intuition}
  \input{figs/plant_developmental_perturbation.tex}
\end{frame}

\begin{frame}{Complete Model}
\input{figs/complete_schematic.tex}
\end{frame}

\begin{frame}{Biological Perspective: Intraindividual Degeneracy}
  idea: employing a diverse collection of substructures that provide identical or near-identical functionality promote robustness through redundancy while providing many jumping off points for variation through repurposing or elaboration
  \input{figs/intraindividual_degeneracy.tex}
\end{frame}

\begin{frame}{Conway's Game of Life}
\begin{figure}
  \includegraphics[width=0.8\textwidth]{img/gol_icon}
  \captionsetup{singlelinecheck=off,justification=raggedright}
\href{https://www.youtube.com/watch?v=Kzg5is1lgSk}{\caption{Video illustrations of Conway's Game of Life cellular automata in action.}}
\end{figure}
\end{frame}

\begin{frame}{Evidence for Indirect Plasticity}
\begin{figure}
    \centering
    \includegraphics[width=0.8\textwidth]{img/primary_secondary_w02}
 	\captionsetup{singlelinecheck=off,justification=raggedright}
  	\caption{Primary and secondary objective performance of champion individuals evolved with primary and secondary condition/objective pair.}
    \label{fig:ev_w0}
\end{figure}
\end{frame}

\begin{frame}{Evidence for Indirect Plasticity}
\begin{figure}
    \centering
    \includegraphics[width=0.8\textwidth]{img/scatter_indirect}
 	\captionsetup{singlelinecheck=off,justification=raggedright}
  	\caption{Comparison of objective performances of champions evolved with only primary condition/objective pair versus with both primary and secondary condition/objective pairs.}
    \label{fig:es_p0}
\end{figure}
\end{frame}

\begin{frame}{Evolvability as Novel Variation}
  \input{figs/individual_vs_population_evolvability.tex}
\end{frame}

\begin{frame}{Promoting Evolvability: Fitness Niches}
	\input{figs/cppn_images.tex}
\end{frame}

\begin{frame}{Promoting Evolvability: Fitness Niches}
\vfill
	\input{figs/dnn.tex}
    \vfill
\end{frame}

\begin{frame}{Promoting Evolvability: Fitness Niches}
\vspace{2ex}
	\input{figs/niches.tex}
\end{frame}

\begin{frame}{Promoting Evolvability: Fitness Niches}
\vspace{2ex}	
\input{figs/goal_switching.tex}
\end{frame}

\begin{frame}{Promoting Evolvability: Fitness Niches}
	\input{figs/ie_results.tex}
\end{frame}
